%imagem

\begin{figure}[!h]
    \centering
    \includegraphics[width = 11cm]{img/parabola.png}
    \caption{Gráfico de Função Parábola (unitária) com $\alpha = \frac{1}{2}$ no intervalo $[0,10]$}
    \label{fig:rampa}
    \begin{flushright}
        Confeccionado pelos autores (2020).
    \end{flushright}
\end{figure} 

%equação varias linhas

\begin{eqnarray*} \label{eq:py}
    P_y &=& \lim_{t\to\infty} \frac{1}{T} \int_{-\frac{T}{2}}^{\frac{T}{2}} [y(t)]^2 dt = \lim_{t\to\infty} \frac{1}{T} \int_{-\frac{T}{2}}^{\frac{T}{2}} [A_1.\cos{(\omega_1.t + \theta_1)}+A_2.\cos{(\omega_2.t + \theta_2)}]^2 dt \\
     &=& \lim_{t\to\infty} \frac{1}{T} \int_{-\frac{T}{2}}^{\frac{T}{2}} A_1^2 \cos^2(\omega_1.t+\theta_1) dt + \lim_{t\to\infty} \frac{1}{T} \int_{-\frac{T}{2}}^{\frac{T}{2}} A_2^2 \cos^2(\omega_2.t+\theta_2) dt \\
    && + \lim_{t\to\infty} \frac{2A_1A_2}{T} \int_{-\frac{T}{2}}^{\frac{T}{2}}  \cos(\omega_1t+\theta_1) \cos(\omega_2t + \theta_2) dt \\
     &=& \lim_{t\to\infty} \frac{1}{T} \int_{-\frac{T}{2}}^{\frac{T}{2}} A_1^2 \cos^2(\omega_1.t+\theta_1) dt + \lim_{t\to\infty} \frac{1}{T} \int_{-\frac{T}{2}}^{\frac{T}{2}} A_2^2 \cos^2(\omega_2.t+\theta_2) dt \\
    &=& \lim_{t\to\infty} \frac{A_1^2}{T} \int_{-\frac{T}{2}}^{\frac{T}{2}} \cos^2(\omega_1.t+\theta_1) dt + \lim_{t\to\infty} \frac{A_2^2}{T} \int_{-\frac{T}{2}}^{\frac{T}{2}} \cos^2(\omega_2.t+\theta_2) dt 
\end{eqnarray*}

%codigo fonte
\begin{lstlisting}[language=Octave]


\end{lstlisting}


% código fonte _____________________________________________________
\usepackage{listings}           % Ambiente para código fonte

\definecolor{codegreen}{rgb}{0,0.6,0}
\definecolor{codegray}{rgb}{0.5,0.5,0.5}
\definecolor{codepurple}{rgb}{0.58,0,0.82}
\definecolor{backcolour}{rgb}{0.95,1,0.95}

\lstdefinestyle{default}{
    backgroundcolor=\color{backcolour},   
    commentstyle=\color{codegreen},
    keywordstyle=\color{magenta},
    numberstyle=\tiny\color{codegray},
    stringstyle=\color{codepurple},
    basicstyle=\ttfamily\footnotesize,
    breakatwhitespace=false,         
    breaklines=true,                 
    captionpos=b,                    
    keepspaces=true,                 
    numbers=left,                    
    numbersep=5pt,                  
    showspaces=false,                
    showstringspaces=false,
    showtabs=false,                  
    tabsize=2
}
\lstset{style=default}

%__________________________________________________________________

%sistema linear
\begin{equation*}
     \begin{cases} 
        x+2y+z = 8\\ 
        2x-y+z=3\\ 
        3x+y-z=2 
    \end{cases}   
\end{equation*}

%referencia livro

@book{thelma,
  title={Comunicação e linguagem},
  author={Thelma de Carvalho GUIMARÃES},
  year={2012},
  address = "Sao Paulo",
  publisher={Pearson Education do Brasil},
  % url = ,
  
}
